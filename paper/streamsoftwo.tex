\documentclass[modern, letterpaper]{aastex61}
% \documentclass[twocolumn]{aastex61}

% TODO:
% -

% style notes
% -----------
% - Use the Makefile; don't be typing ``pdflatex'' or some bullshit.
% - Line break between sentences to make the git diffs readable.
% - Use \, as a multiply operator.
% - Reserve () for function arguments; use [] or {} for outer shit.
% - Use \sectionname not Section, \figname not Figure, \documentname not Article

\include{gitstuff}
\include{preamble}
\graphicspath{{figures/}}
\usepackage{microtype}

% Project-specific
\newcommand{\gaia}{\project{Gaia}}

\newcommand{\tub}{\ensuremath{t_{\rm ub}}}

\begin{document}\sloppy\sloppypar\raggedbottom\frenchspacing % trust me

\title{Stellar streams of two stars}

\author{Adrian~M.~Price-Whelan}
\affiliation{Department of Astrophysical Sciences,
             Princeton University, Princeton, NJ 08544, USA}

\author{+++}

% \author{David~N.~Spergel}
% \affiliation{Flatiron Institute,
%              Simons Foundation,
%              162 Fifth Avenue,
%              New York, NY 10010, USA}
% \affiliation{Department of Astrophysical Sciences,
%              Princeton University, Princeton, NJ 08544, USA}

\correspondingauthor{Adrian M. Price-Whelan}
\email{adrn@astro.princeton.edu}

\begin{abstract}
% Context
% Aims
% Methods
% Results
% Conclusions
\end{abstract}

\keywords{
}

\section{Introduction}\label{sec:introduction}


\section{TODO} %\label{sec:introduction}

Our goal is to predict the steady-state distribution of separations and other
dynamical quantities for unbound comoving star pairs (i.e. former wide binaries
and open cluster stars) in the Galactic disk.
To do this, we will make several simplifying assumptions so that we can
analytically incorporate perturbative effects from a given population of
pertubing masses (e.g., giant molecular clouds, small-scale dark matter
subhalos, black holes).
In particular, we will work in angle-action coordinates and assume that the
encounters between the perturbers and a given comoving pair are impulsive so
that the perturbations add linearly.
Below, we briefly describe the formalism for computing the final state of a
single, unbound comoving pair.

We will assume that the star pairs all come unbound once the separation is
larger than the tidal (Jacobi) radius of the binary, and assume that the wide
binaries are all on circular orbits with isotropically distributed angular
momentum vectors.
\todo{TODO: assumptions about background potential}
Given a distribution function for stellar orbits in the disk, $f_*(\bs{J})$, the
assumption that the stellar orbits are well-mixed so that the angle distribution
is uniform, $p(\bs{\theta}) = \mathcal{U}(\bs{\theta})$, and a joint
distribution of binary component stellar masses, $p(m_1, m_2)$, we can therefore
compute the initial separation and velocity difference between the two stars in
the comoving pair, $(\Delta \bs{x}, \Delta \bs{v})$,\footnote{We use $\Delta$ to
indicate differences in quantities, typically between two stars in a comoving
pair. Later, we use $\delta$ to indicate a small deviation or offset in a
quantity.} for samples from these distributions.
In detail, we sample a single set of orbital actions, $\bs{J}_1 \sim
f_*(\bs{J})$ and angles $\bs{\theta}_1 \sim p(\bs{\theta})$, and two stellar
masses $(m_1, m_2) \sim p(m_1, m_2)$, convert the actions to phase-space
coordinates $(\bs{x}, \bs{v})_1$ assuming a uniform angle distribution, and
compute the Jacobi radius as
\begin{equation}
r_{\rm J} = \left[\frac{G\,(m_1 + m_2)}{4\,\Omega_g \, A_g} \right]^{1/3}\quad .
\end{equation}
Assuming that the angular momenta of the wide binaries prior to disruption are
isotropically distributed and circular, we can then compute the phase-space
coordinates of the companion, $(\bs{x}, \bs{v})_2$, at the time of disruption,
corresponding to action-angle coordinates $(\bs{J}, \bs{\theta})_2$.
- Actions have corresponding frequencies, will work in freq.-angle space
- Start with initial separation in frequency, angle
- Unperturbed case, angle difference grows linearly along preferred direction
- Perturbed case, need to add effects of perturbations
- Given a time of perturbation, update angle and frequency as:

\begin{eqnarray}
\Delta\bs{\theta} &=& \Delta\bs{\theta}_0 +
  \Delta\bs{\Omega}_0 \, (t - \tub) + \delta\bs{\theta} +
  \delta\bs{\Omega} \, (t - \tub - t^*)\\
\Delta\bs{\Omega} &=& \Delta\bs{\Omega}_0 + \delta\bs{\Omega}
\end{eqnarray}

where

\begin{eqnarray}
\delta\bs{\theta} &=& \left.\frac{\partial\bs{\theta}}{\partial\bs{v}}\right|_{\bs{x}} \cdot \delta\bs{v}^g\\
\delta\bs{\Omega} &=& \left.\frac{\partial\bs{\Omega}}{\partial\bs{v}}\right|_{\bs{x}} \cdot \delta\bs{v}^g\\
\end{eqnarray}

TODO: why g superscript???

\acknowledgements

% This work has made use of data from the European Space Agency (ESA)
% mission {\it Gaia} (\url{https://www.cosmos.esa.int/gaia}), processed by
% the {\it Gaia} Data Processing and Analysis Consortium (DPAC,
% \url{https://www.cosmos.esa.int/web/gaia/dpac/consortium}). Funding
% for the DPAC has been provided by national institutions, in particular
% the institutions participating in the {\it Gaia} Multilateral Agreement.
% This research has made use of the SIMBAD database, operated at CDS, Strasbourg,
% France.
% The authors are pleased to acknowledge that the work reported on in this
% paper was substantially performed at the TIGRESS high performance computer
% center at Princeton University which is jointly supported by the Princeton
% Institute for Computational Science and Engineering and the Princeton
% University Office of Information Technology's Research Computing department.

\software{
The code used in this project is available from
\url{https://github.com/adrn/StreamsOfTwo} under the MIT open-source
software license.
% This research utilized the following open-source \python\ packages:
%     \package{Astropy} (\citealt{Astropy-Collaboration:2013}),
%     \package{astroquery} (\citealt{Ginsburg:2016}),
%     \package{ccdproc} (\citealt{Craig:2015}),
%     \package{celerite} (\citealt{Foreman-Mackey:2017}),
%     \package{corner} (\citealt{Foreman-Mackey:2016}),
%     \package{emcee} (\citealt{Foreman-Mackey:2013ascl}),
%     \package{IPython} (\citealt{Perez:2007}),
%     \package{matplotlib} (\citealt{Hunter:2007}),
%     \package{numpy} (\citealt{Van-der-Walt:2011}),
%     \package{scipy} (\url{https://www.scipy.org/}),
%     \package{sqlalchemy} (\url{https://www.sqlalchemy.org/}).
% This work additionally used the Gaia science archive
% (\url{https://gea.esac.esa.int/archive/}), and the SIMBAD database
% (\citealt{Wenger:2000}).
}

% \facility{MDM: Hiltner (Modspec)}

\bibliographystyle{aasjournal}
\bibliography{refs}

\end{document}
