\documentclass[modern, letterpaper]{aastex61}
% \documentclass[twocolumn]{aastex61}

% TODO:
% -

% style notes
% -----------
% - Use the Makefile; don't be typing ``pdflatex'' or some bullshit.
% - Line break between sentences to make the git diffs readable.
% - Use \, as a multiply operator.
% - Reserve () for function arguments; use [] or {} for outer shit.
% - Use \sectionname not Section, \figname not Figure, \documentname not Article

\include{gitstuff}
% Load common packages

\usepackage{amsmath}
\usepackage{amsfonts}
\usepackage{amssymb}
\usepackage{booktabs}

\usepackage{graphicx}
\usepackage{color}

\usepackage{hyperref}
\definecolor{linkcolor}{rgb}{0.02,0.35,0.55}
\definecolor{citecolor}{rgb}{0.4,0.4,0.4}
\hypersetup{colorlinks=true,linkcolor=linkcolor,citecolor=citecolor,
            filecolor=linkcolor,urlcolor=linkcolor}
\hypersetup{pageanchor=false}

\newcommand{\documentname}{\textsl{Article}}
\newcommand{\sectionname}{Section}
\newcommand{\figname}{Figure}
\newcommand{\eqname}{Equation}
\newcommand{\tblname}{Table}

% Packages / projects / programming
\newcommand{\package}[1]{\texttt{#1}}
\newcommand{\acronym}[1]{{\small{#1}}}
\newcommand{\github}{\package{GitHub}}
\newcommand{\python}{\package{Python}}

% Missions
\newcommand{\project}[1]{\textsl{#1}}

% For referee
\newcommand{\changes}[1]{{\color{red} #1}}

% Stats / probability
\newcommand{\given}{\,|\,}
\newcommand{\norm}{\mathcal{N}}

% Maths
\newcommand{\dd}{\mathrm{d}}
\newcommand{\transpose}[1]{{#1}^{\mathsf{T}}}
\newcommand{\inverse}[1]{{#1}^{-1}}
\newcommand{\argmin}{\operatornamewithlimits{argmin}}
\newcommand{\argmax}{\operatornamewithlimits{argmax}}
\newcommand{\mean}[1]{\left< #1 \right>}

% Unit shortcuts
\newcommand{\msun}{\ensuremath{\mathrm{M}_\odot}}
\newcommand{\kms}{\ensuremath{\mathrm{km}~\mathrm{s}^{-1}}}
\newcommand{\au}{\ensuremath{\mathrm{au}}}
\newcommand{\pc}{\ensuremath{\mathrm{pc}}}
\newcommand{\kpc}{\ensuremath{\mathrm{kpc}}}
\newcommand{\kmskpc}{\ensuremath{\mathrm{km}~\mathrm{s}^{-1}~\mathrm{kpc}^{-1}}}

% Misc.
\newcommand{\bs}[1]{\boldsymbol{#1}}

% Astronomy
\newcommand{\DM}{{\rm DM}}
\newcommand{\feh}{\ensuremath{{[{\rm Fe}/{\rm H}]}}}
\newcommand{\df}{\acronym{DF}}

% TO DO
\newcommand{\todo}[1]{{\color{red} TODO: #1}}

\graphicspath{{figures/}}
\usepackage{microtype}

% Project-specific
\newcommand{\gaia}{\project{Gaia}}
\newcommand{\tub}{\ensuremath{t_{\rm ub}}}
\newcommand{\DF}{\textsc{df}}

\begin{document}\sloppy\sloppypar\raggedbottom\frenchspacing % trust me

\title{Stellar streams of two stars}

\author{Adrian~M.~Price-Whelan}
\affiliation{Department of Astrophysical Sciences,
             Princeton University, Princeton, NJ 08544, USA}

\author{others}

% \author{David~N.~Spergel}
% \affiliation{Flatiron Institute,
%              Simons Foundation,
%              162 Fifth Avenue,
%              New York, NY 10010, USA}
% \affiliation{Department of Astrophysical Sciences,
%              Princeton University, Princeton, NJ 08544, USA}

\correspondingauthor{Adrian M. Price-Whelan}
\email{adrn@astro.princeton.edu}

\begin{abstract}
% Context
% Aims
% Methods
% Results
% Conclusions
\end{abstract}

\keywords{
}

\section{Introduction}\label{sec:introduction}


\section{TODO} %\label{sec:introduction}

Our goal is to predict the steady-state distribution of separations and other
dynamical quantities for unbound comoving star pairs (i.e. former wide binaries
and open cluster stars) in the Galactic disk.
To do this, we will make several simplifying assumptions so that we can
analytically incorporate perturbative effects from a given population of
pertubing masses (e.g., giant molecular clouds, small-scale dark matter
subhalos, black holes).
In particular, we will work in angle-action coordinates and assume that the
encounters between the perturbers and a given comoving pair are impulsive so
that the perturbations add linearly.
Below, we briefly describe the formalism for computing the final state of a
single, unbound comoving pair.

We will assume that the star pairs all come unbound once the separation is
larger than the tidal (Jacobi) radius of the binary, and assume that the wide
binaries are all on circular orbits with isotropically distributed angular
momentum vectors.
\todo{TODO: assumptions about background potential}
Given a distribution function, \DF, for stellar orbits in the disk,
$f_*(\bs{J})$, the assumption that the stellar orbits are well-mixed so that the
angle distribution is uniform, $p(\bs{\theta}) = \mathcal{U}(\bs{\theta})$, and
a joint distribution of binary component stellar masses, $p(m_1, m_2)$, we can
therefore compute the initial separation and velocity difference between the two
stars in the comoving pair, $(\Delta \bs{x}, \Delta \bs{v})$,\footnote{We use
$\Delta$ to indicate differences in quantities, typically between two stars in a
comoving pair. Later, we use $\delta$ to indicate a small deviation or offset in
a quantity.} for samples from these distributions.
In detail, we sample a single set of orbital actions, $\bs{J}_1 \sim
f_*(\bs{J})$ and angles $\bs{\theta}_1 \sim p(\bs{\theta})$, and two stellar
masses $(m_1, m_2) \sim p(m_1, m_2)$, convert the actions to phase-space
coordinates $(\bs{x}, \bs{v})_1$ assuming a uniform angle distribution, and
compute the Jacobi radius as
\begin{equation}
    r_{\rm J} = \left[\frac{G\,(m_1 + m_2)}{4\,\Omega_g \, A_g} \right]^{1/3}\quad .
\end{equation}
Assuming that the angular momenta of the wide binaries prior to disruption are
isotropically distributed and circular, we can then compute the phase-space
coordinates of the companion, $(\bs{x}, \bs{v})_2$, at the time of disruption,
corresponding to action-angle coordinates $(\bs{J}, \bs{\theta})_2$.
The actions, $\bs{J}$, have corresponding frequencies
\begin{equation}
    \bs{\Omega} = \frac{\partial H}{\partial \bs{J}}
\end{equation}
that are also constants of motion as both $H$ and $\bs{J}$ are dynamical
invariants.
Following previous work, we will use frequencies in place of actions, so that
the Galactic orbits of each star in a comoving pair are described by
$(\bs{\Omega}, \bs{\theta})$.

With the assumptions above, we start with the frequencies and angles of each
star in a comoving pair, $(\bs{\Omega}, \bs{\theta})_1$ and $(\bs{\Omega},
\bs{\theta})_2$, at the unbinding time, \tub, of the initial wide binary.
In a smooth background potential, the difference in frequencies between the two
stars is fixed, but the difference in angle variables increases linearly with
time, $t$,
\begin{eqnarray}
    \Delta\bs{\theta}(t) &=& \Delta\bs{\theta}_0 + \Delta\bs{\Omega}_0 \, (t - \tub)\\
    \Delta\bs{\Omega} &=& \Delta\bs{\Omega}_0 = {\rm const.}
\end{eqnarray}
where $\Delta\bs{\theta}_0$ is the initial separation in angles and
$\Delta\bs{\Omega}_0$ is the initial separation in frequencies, at the time of
unbinding.
This shearing is what leads to the growing physical separation between the pair
and occurs preferentially in a direction determined by a projection of the
largest eigenvector of the Hessian matrix, $\frac{\partial^2 H}{\partial
\bs{J}_i \bs{J}_j}$, into configuration space (\citealt{Sanders:2013}).

When the background gravitational potential contains a population of massive
perturbers, encounters between the comoving pair and a given perturber will
impart changes to the angles and frequencies of the two stars.
If we invoke the impulse approximation and assume that the duration of any given
encounter is much shorter than an orbit --- i.e. ignore the curvature of the
orbits --- then the encounter only induces a net velocity kick, $\delta \bs{v}$.
This (instantaneously applied) velocity kick can be expressed as changes to the
angles and frequencies
\begin{eqnarray}
    \delta\bs{\theta} &=& \left.\frac{\partial\bs{\theta}}
        {\partial\bs{v}}\right|_{\bs{x}} \cdot \delta\bs{v}\label{eq:ang-dv}\\
    \delta\bs{\Omega} &=& \left.\frac{\partial\bs{\Omega}}
        {\partial\bs{v}}\right|_{\bs{x}} \cdot \delta\bs{v}\label{eq:freq-dv}
\end{eqnarray}
where the Jacobian matrices above generally have to be computed numerically
(from here on, we follow a similar prescription to that used by
\citealt{Sanders:2016}).
The difference in angles and frequencies between the two stars after the
encounter at time $t^*$ are then
\begin{eqnarray}
\Delta\bs{\theta}(t) &=& \Delta\bs{\theta}_0 +
    \Delta\bs{\Omega}_0 \, (t - \tub) + \delta\bs{\theta} +
    \delta\bs{\Omega} \, (t - \tub - t^*)\\
\Delta\bs{\Omega} &=& \Delta\bs{\Omega}_0 + \delta\bs{\Omega} \quad .
\end{eqnarray}

A given pair could encounter multiple perturbers and therefore receive several
kicks at times $t^{(k)}$.
In these coordinates, the pertubations add linearly so that
\begin{eqnarray}
\Delta\bs{\theta}(t) &=& \Delta\bs{\theta}_0 +
    \Delta\bs{\Omega}_0 \, (t - \tub) + \sum_k \left[ \delta\bs{\theta}^{(k)} +
    \delta\bs{\Omega}^{(k)} \, (t - \tub - t^{(k)}) \right]\\
\Delta\bs{\Omega} &=& \Delta\bs{\Omega}_0 + \delta\bs{\Omega}^{(k)} \quad .
\end{eqnarray}
Therefore, given a set of encounter times, perturber orbital properties, and
the masses or internal structure of the perturbers, we can compute the
separation in angles and frequencies between the two stars at any time, $t$.

The perturber masses may have a different \DF, $f_{\rm p}(\bs{J})$, and may have
different internal properties --- e.g., mass, scale-length --- relative to the
star pair components themselves.
For example, giant molecular clouds (GMCs) will have large mass, internal
scale-lengths comparable to the comoving pair separation, and orbits with a
smaller vertical scale-height and radial scale-length in the Galaxy.
For simplicity, we assume that the perturber velocity distribution relative to
the stellar velocity distribution is Gaussian.
We then generate the parameters of a particular encounter, $k$, at time
$t^{(k)}$, using the following procedure:
\begin{enumerate}
    \item Compute the current state of the pair in configuration-space --- i.e.
        $(\bs{x},\bs{v})_1$, $(\bs{x},\bs{v})_2$ --- at time $t^{(k)}$;
    \item Sample an impact parameter, $b$, from a uniform distribution
        $\mathcal{U}(-b_{\rm max}, b_{\rm max})$ following previous work
        (\citealt{Bovy:2017}) --- the exact value of $b_{\rm max}$ is defined
        later;
    \item Sample the perturber velocity relative to the star pair at the given
        impact parameter assuming that the encounter is a straight line and the
        orientation of the encounter is isotropically distributed;
    \item Finally, we compute $\delta\bs{v}^{(k)}$ and convert this kick to
        changes in angle and frequency using \eqname
        s~\ref{eq:ang-dv}--\ref{eq:freq-dv}.
\end{enumerate}

To generate encounter times for a given pair of stars at any time, we estimate
the cross-section for interaction with a given perturber, which depends on the
instantaneous separation of the pair.
We then sample an encounter time from a Poisson distribution with the given
rate.
After updating the angles and frequencies (and therefore positions and
velocities) of the pair, we then repeat this procedure until we reach the end
time of the simulation for a single pair.
We then convert the angle and frequency separation of the pair back to
configuration space.

By repeating this procedure for a sample of comoving pairs, we can then predict
the steady-state separation distribution of unbound comoving star pairs in the
Galactic disk.

 - Also: velocity difference distribution, on-sky correlation function...


\acknowledgements

% This work has made use of data from the European Space Agency (ESA)
% mission {\it Gaia} (\url{https://www.cosmos.esa.int/gaia}), processed by
% the {\it Gaia} Data Processing and Analysis Consortium (DPAC,
% \url{https://www.cosmos.esa.int/web/gaia/dpac/consortium}). Funding
% for the DPAC has been provided by national institutions, in particular
% the institutions participating in the {\it Gaia} Multilateral Agreement.
% This research has made use of the SIMBAD database, operated at CDS, Strasbourg,
% France.
% The authors are pleased to acknowledge that the work reported on in this
% paper was substantially performed at the TIGRESS high performance computer
% center at Princeton University which is jointly supported by the Princeton
% Institute for Computational Science and Engineering and the Princeton
% University Office of Information Technology's Research Computing department.

\software{
The code used in this project is available from
\url{https://github.com/adrn/StreamsOfTwo} under the MIT open-source
software license.
% This research utilized the following open-source \python\ packages:
%     \package{Astropy} (\citealt{Astropy-Collaboration:2013}),
%     \package{astroquery} (\citealt{Ginsburg:2016}),
%     \package{ccdproc} (\citealt{Craig:2015}),
%     \package{celerite} (\citealt{Foreman-Mackey:2017}),
%     \package{corner} (\citealt{Foreman-Mackey:2016}),
%     \package{emcee} (\citealt{Foreman-Mackey:2013ascl}),
%     \package{IPython} (\citealt{Perez:2007}),
%     \package{matplotlib} (\citealt{Hunter:2007}),
%     \package{numpy} (\citealt{Van-der-Walt:2011}),
%     \package{scipy} (\url{https://www.scipy.org/}),
%     \package{sqlalchemy} (\url{https://www.sqlalchemy.org/}).
% This work additionally used the Gaia science archive
% (\url{https://gea.esac.esa.int/archive/}), and the SIMBAD database
% (\citealt{Wenger:2000}).
}

% \facility{MDM: Hiltner (Modspec)}

\bibliographystyle{aasjournal}
\bibliography{refs}

\end{document}
